\documentclass{beamer}
\usepackage{graphicx}
\usepackage{booktabs}
\usepackage{amsmath}
\usepackage{amssymb}

\usetheme{Madrid}
\usecolortheme{whale}

\title{Analisi Statistica del Dataset Heart Disease UCI}
\author{Il Tuo Nome}
\institute{Dipartimento di Statistica, Università XYZ}
\date{\today}

\begin{document}

\frame{\titlepage}

\begin{frame}
\frametitle{Indice}
\tableofcontents
\end{frame}

\section{Introduzione}

\begin{frame}
\frametitle{Introduzione}
\begin{itemize}
    \item \textbf{Obiettivo}: Analizzare il dataset Heart Disease UCI per identificare pattern e correlazioni significative
    \item \textbf{Dataset}: 303 osservazioni, 14 variabili
    \item \textbf{Variabile target}: Presenza di malattia cardiaca (num)
    \item \textbf{Metodologia}: Analisi esplorativa dei dati (EDA) utilizzando R
    \item \textbf{Focus}:
    \begin{itemize}
        \item Preparazione e pulizia dei dati
        \item Analisi statistica descrittiva
        \item Visualizzazione delle distribuzioni
        \item Identificazione di potenziali predittori di malattia cardiaca
    \end{itemize}
\end{itemize}
\end{frame}

\section{Metodologia}

\begin{frame}
\frametitle{Metodologia}
\begin{enumerate}
    \item \textbf{Caricamento e ispezione dei dati}:
    \begin{itemize}
        \item Utilizzo di \texttt{read\_csv()} per importare il dataset
        \item Esame della struttura con \texttt{str()} e \texttt{head()}
    \end{itemize}
    \item \textbf{Gestione dei valori mancanti}:
    \begin{itemize}
        \item Identificazione con \texttt{is.na()}
        \item Imputazione KNN con la funzione \texttt{kNN()} del pacchetto VIM
    \end{itemize}
    \item \textbf{Analisi statistica descrittiva}:
    \begin{itemize}
        \item Calcolo di statistiche riepilogative per variabili numeriche
        \item Analisi di frequenza per variabili categoriche
    \end{itemize}
    \item \textbf{Visualizzazione dei dati}:
    \begin{itemize}
        \item Utilizzo di ggplot2 per creare grafici informativi
    \end{itemize}
\end{enumerate}
\end{frame}

\section{Preparazione dei Dati}

\begin{frame}
\frametitle{Preparazione dei Dati}
\begin{itemize}
    \item \textbf{Conversione dei tipi di dati}:
    \begin{itemize}
        \item Variabili categoriche convertite in fattori: \texttt{ca}, \texttt{num}
    \end{itemize}
    \item \textbf{Gestione dei valori mancanti}:
    \begin{itemize}
        \item Utilizzo dell'imputazione KNN:
        \[\text{KNN}_k(x) = \frac{1}{k}\sum_{i \in N_k(x)} y_i\]
        dove $N_k(x)$ sono i $k$ vicini più prossimi di $x$
    \end{itemize}
    \item \textbf{Verifica dell'integrità dei dati}:
    \begin{itemize}
        \item Controllo dei range e delle distribuzioni delle variabili
    \end{itemize}
\end{itemize}
\end{frame}

\section{Analisi Statistica}

\begin{frame}
\frametitle{Analisi Statistica - Misure di Tendenza Centrale e Dispersione}
Per ogni variabile numerica $X$, calcoliamo:
\begin{itemize}
    \item \textbf{Media}: $\bar{X} = \frac{1}{n}\sum_{i=1}^n X_i$
    \item \textbf{Mediana}: valore centrale della distribuzione ordinata
    \item \textbf{Deviazione Standard}: $s = \sqrt{\frac{1}{n-1}\sum_{i=1}^n (X_i - \bar{X})^2}$
    \item \textbf{Varianza}: $s^2 = \frac{1}{n-1}\sum_{i=1}^n (X_i - \bar{X})^2$
    \item \textbf{Range Interquartile (IQR)}: $IQR = Q_3 - Q_1$
\end{itemize}
\end{frame}

\begin{frame}
\frametitle{Analisi Statistica - Misure di Forma}
Per valutare la forma delle distribuzioni:
\begin{itemize}
    \item \textbf{Skewness}: 
    \[g_1 = \frac{m_3}{m_2^{3/2}} = \frac{\frac{1}{n}\sum_{i=1}^n (X_i - \bar{X})^3}{s^3}\]
    dove $m_k$ è il k-esimo momento centrale
    \item \textbf{Kurtosis}: 
    \[g_2 = \frac{m_4}{m_2^2} - 3 = \frac{\frac{1}{n}\sum_{i=1}^n (X_i - \bar{X})^4}{s^4} - 3\]
\end{itemize}
\end{frame}

\section{Visualizzazione dei Dati}

\begin{frame}
\frametitle{Visualizzazione dei Dati - Variabili Quantitative}
Per ogni variabile quantitativa:
\begin{itemize}
    \item \textbf{Istogramma con curva di densità normale}:
    \begin{itemize}
        \item Funzione di densità normale: 
        \[f(x) = \frac{1}{\sigma\sqrt{2\pi}} e^{-\frac{1}{2}(\frac{x-\mu}{\sigma})^2}\]
    \end{itemize}
    \item \textbf{Boxplot}: per identificare outliers e asimmetrie
    \item \textbf{Q-Q plot}: per valutare la normalità della distribuzione
\end{itemize}
\end{frame}

\begin{frame}
\frametitle{Visualizzazione dei Dati - Variabili Categoriche}
Per le variabili categoriche:
\begin{itemize}
    \item \textbf{Diagrammi a barre}: per visualizzare le frequenze relative
    \item \textbf{Tabelle di contingenza}: per esaminare le relazioni tra variabili categoriche
    \item \textbf{Test Chi-quadrato}: per valutare l'indipendenza tra variabili categoriche
    \[\chi^2 = \sum_{i=1}^r \sum_{j=1}^c \frac{(O_{ij} - E_{ij})^2}{E_{ij}}\]
    dove $O_{ij}$ sono le frequenze osservate e $E_{ij}$ quelle attese
\end{itemize}
\end{frame}

\section{Risultati Preliminari}

\begin{frame}
\frametitle{Risultati Preliminari}
\begin{itemize}
    \item \textbf{Distribuzione delle malattie cardiache}:
    \begin{itemize}
        \item Percentuale di pazienti con malattia cardiaca
        \item Differenze di genere nella prevalenza
    \end{itemize}
    \item \textbf{Correlazioni significative}:
    \begin{itemize}
        \item Matrice di correlazione per variabili numeriche
        \item Identificazione di potenziali predittori
    \end{itemize}
    \item \textbf{Analisi delle variabili chiave}:
    \begin{itemize}
        \item età, pressione sanguigna, colesterolo, ecc.
    \end{itemize}
\end{itemize}
\end{frame}

\section{Conclusioni e Prospettive Future}

\begin{frame}
\frametitle{Conclusioni e Prospettive Future}
\begin{itemize}
    \item \textbf{Riepilogo delle principali scoperte}:
    \begin{itemize}
        \item Pattern identificati nei dati
        \item Variabili più correlate con la presenza di malattia cardiaca
    \end{itemize}
    \item \textbf{Implicazioni cliniche}:
    \begin{itemize}
        \item Potenziali fattori di rischio emergenti
        \item Suggerimenti per la pratica clinica
    \end{itemize}
    \item \textbf{Limitazioni dello studio}:
    \begin{itemize}
        \item Dimensione del campione
        \item Possibili bias nei dati
    \end{itemize}
    \item \textbf{Direzioni future}:
    \begin{itemize}
        \item Analisi inferenziale approfondita
        \item Modelli predittivi (es. regressione logistica, machine learning)
    \end{itemize}
\end{itemize}
\end{frame}

\begin{frame}
\frametitle{Grazie per l'attenzione}
\begin{center}
    \large{Domande?}
\end{center}
\end{frame}

\end{document}